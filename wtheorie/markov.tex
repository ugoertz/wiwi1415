\documentclass{ximera}
% \input{preamble.tex}
\usepackage[utf8]{inputenc}
\title{Endliche Markovketten: Ein Beispiel}
\begin{document}
\begin{abstract}
Endliche Markovketten in einem Anwendungsbeispiel.
\end{abstract}
\maketitle

In diesem Beispiel lernen wir das Konzept der Markovkette aus der Wahrscheinlichkeitstheorie kennen. Technisches Hilfsmittel ist Matrizenrechnung.

Wir betrachten Personen, die jeweils \emph{beschäftigt} (B) oder \emph{arbeitssuchend} (A) sind. Zu jedem Monatsersten kann sich der Status ändern, d.h. manche Arbeitssuchende finden Beschäftigung und umgekehrt. Und zwar ist
\begin{align*}
& \text{die Wahrscheinlichkeit, dass ein Beschäftigter beschäftigt bleibt:} p_{BB} = 0.9 \\
& \text{die Wahrscheinlichkeit, dass ein Beschäftigter arbeitslos wird:} p_{BA} = 0.1 \\
& \text{die Wahrscheinlichkeit, dass ein Arbeitssuchender arbeitslos bleibt:} p_{AA} = 0.3 \\
\end{align*}


\begin{question}
Wie groß ist die Wahrscheinlichkeit, dass ein Arbeitssuchender Beschäftigung findet?
\begin{solution}
\begin{hint}
Die Wahrscheinlichkeit muss sich mit $p_{AA}$ zu $1$ aufaddieren.
\end{hint}
\begin{hint}
Es ist also $p_{AB} = 1 - p_{AA} = 0.7$.
\end{hint}
Es ist $p_{AB} =$ \answer{0.7}.
\end{solution}
\end{question}


Nehmen wir nun an, dass aufgrund einer Werksschließung $1200$ Personen entlassen werden und arbeitssuchend sind. Aufgrund der oben angegebenen Wahrscheinlichkeiten erwarten wir, dass nach einem Monat ca.~$0.7 \cdot 1200 = 840$ Personen Arbeit gefunden haben und $0.3\cdot 1200 = 360$ Personen noch arbeitssuchend sind.

\begin{question}
Wie ist es nun nach zwei Monaten?
\begin{solution}
\begin{hint}
Beachte, dass sowohl ein teil der Beschäftigten beschäftigt bleibt (nämlich 90\%), als auch ein Teil der Arbeitssuchenden Arbeit finden wird (nämlich 70\%).
\end{hint}
\begin{hint}
Zu berechnen ist also $0.9 \cdot 840 + 0.7 \cdot 360$.
\end{hint}
Beschäftigt sind \answer{1008} Personen.
\end{solution}
\begin{solution}
\begin{hint}
Es gibt zwei Möglichkeiten, die Antwort zu berechnen. Eine analoge Rechnung wie im ersten Fall ist $0.1 \cdot 840 + 0.3 \cdot 360$. Da aber die Arbeitssuchenden genau diejenigen Personen sind, die nicht beschäftigt sind, lässt sich die Antwort auch aus dem ersten Teil berechnen als $1200 - 1008$.
\end{hint}
Arbeitssuchend sind \answer{192} Personen.
\end{solution}
\end{question}


Die Ausdrücke $0.9 \cdot 840 + 0.7 \cdot 360$ und $0.1 \cdot 840 + 0.3 \cdot 360$ können wir elegant als das Ergebnis einer Matrizenmultiplikation zusammenfassen:
\[
\left(
\begin{array}{cc}
0.9 & 0.7 \\
0.1 & 0.3    
\end{array}
\right) \left(
\begin{array}{c}
840 \\ 360    
\end{array}
\right) = \left(
\begin{array}{c}
0.9 \cdot 840 + 0.7 \cdot 360 \\
0.1 \cdot 840 + 0.3 \cdot 360
\end{array}
\right) = \left(
\begin{array}{c}
1008 \\ 192
\end{array}
\right).
\]

Die entsprechende Rechnung funktioniert natürlich auch für den ersten Schritt:
\[
\left(
\begin{array}{cc}
0.9 & 0.7 \\
0.1 & 0.3    
\end{array}
\right) \left(
\begin{array}{c}
0 \\ 1200    
\end{array}
\right) = \left(
\begin{array}{c}
0.9 \cdot 0 + 0.7 \cdot 1200 \\
0.1 \cdot 0 + 0.3 \cdot 1200
\end{array}
\right) = \left(
\begin{array}{c}
840 \\ 360
\end{array}
\right).
\]

Als allgemeine Formel erhalten wir für den Zustand im $n$-ten Monat, geschrieben als Spaltenvektor wie oben:
\[
\left(
\begin{array}{cc}
0.9 & 0.7 \\
0.1 & 0.3    
\end{array}
\right)^n \cdot \left(
\begin{array}{c}
0 \\ 1200    
\end{array}
\right).
\]

Die Matrix 
\[
\left(
\begin{array}{cc}
p_{BB} & p_{AB} \\
p_{BA} & p_{AA}    
\end{array}
\right) = 
\left(
\begin{array}{cc}
0.9 & 0.7 \\
0.1 & 0.3    
\end{array}
\right)
\]
heißt die \emph{Übergangsmatrix}.

Dass die Aufteilung zwischen beschäftigt und arbeitssuchend in jedem Monat nur von der Aufteilung im Vormonat, nicht aber von weiter zurückliegenden Monaten abhängt, nennt man die \emph{Markov-Eigenschaft}.


Was bedeutet es, wenn die Übergangsmatrix gleiche Spalten hat?

Potenzen der Übergangsmatrix
 .. sind wieder Übergangsmatrizen
 .. Konvergenzfrage



\textbf{Literatur:} Chiang, Wainwright, Nitsch, Mathematik für Ökonomen, Kapitel 4.7.

\end{document}
