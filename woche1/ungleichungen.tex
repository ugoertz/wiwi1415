\documentclass{ximera}
% \input{preamble.tex}
\usepackage[utf8]{inputenc}
\title{Woche 1/Ungleichungen}
\begin{document}
\begin{abstract}
Hier finden Sie eine Diskussion der Aufgabe 4b auf dem ersten Übungsblatt.
\end{abstract}
\maketitle


Die Aufgabe 4b war, zu überprüfen, ob für alle reellen Zahlen $x\ge 0$ die Ungleichung $x^3 + \frac 13 \ge x$ gilt.

Sofern $x\ge 1$ ist, ist $x^3 \ge x$, also erst recht $x^3 + \frac 13 \ge x$. Ein Problem kann sich also höchstens für $x$ mit $o\le x\le 1$ ergeben.

\begin{question}
Und tatsächlich: Werten wir die linke Seite aus für $x = \frac 12$, so erhalten wir:
\begin{solution}
\begin{hint}
Es ist $\left(\frac 12\right)^3 = \frac 12 \cdot \frac 12\cdot \frac 12 = \frac 18$. Bringen Sie nun beide Brüche auf den selben Nenner.
\end{hint}
Es ist $\left(\frac 12\right)^3 + \frac 13=$ \answer{$11/24$}
\end{solution}
\begin{solution}
Wir sehen also: Für $x= \frac 12$ ist die obige Ungleichung nicht erfüllt, denn $\frac{11}{24}< \frac{12}{24} = \frac 12$.
\end{solution}
\end{question}


Wie ist es stattdessen mit der Ungleichung $x^3 + 1 \ge x$. Da wir die linke Seite nun vergrößert haben, sind die Aussichten sicherlich besser, dass die Ungleichung erfüllt ist: Und tatsächlich, für $x\ge 1$ ist ---wie oben--- $x^3\ge x$, also erst recht $x^3+1\ge x$. Und für $x\le 1$ ist $1\ge x$, also erst recht (da $x\ge 0$): $x^3+1\ge x$.

\begin{question}
Nun könnten wir versuchen, noch ``bessere'' Ungleichungen zu finden, die zwar für alle $x$ erfüllt sind, wo aber die Konstante auf der linken Seite möglichst klein ist. Wie ist es zum Beispiel mit der Ungleichung $x^3 + \frac 12 \ge x$? Offenbar ist diese erfüllt für $x\ge 1$, und auch für alle $x$ mit $0\le x\le \frac 12$. Werten wir die linke Seite für $x=\frac 34$ aus, so erhalten wir:
\begin{solution}
Es ist $\left(\frac 34\right)^3 + \frac 12 =$ \answer{59/64}.
\end{solution}
\begin{solution}
Also ist die Ungleichung für den Wert $x = \frac 34$ erfüllt.
\end{solution}
\end{question}

In der Tat kann man zeigen, dass für alle $x\ge 0$ die Ungleichung $x^3 + \frac 12 \ge x$ gilt. Eine Möglichkeit ist, die verbleibende Menge aller $x$ zwischen $\frac 12$ und $1$ in kleine Teile einzuteilen und dort jeweils die beiden Seiten abzuschätzen. Zum Beispiel gilt für $x$ mit $\frac 12 \le x \le \frac 58$: Es ist $x^3 \ge \frac 18$, also $x^3 + \frac 12 \ge \frac 58 \ge x$. Nun müsste man noch alle $x$ mit $\frac 58 < x < 1$ in ähnlicher Weise behandeln (man muss dann aber noch kleinere Abschnitte verwenden, was dieses Vorgehen ziemlich lästig macht.)

Einfacher wird es für uns später, wenn wir die Methoden der Differentialrechnung zur Verfügung haben: Dann werden wir zeigen können, dass die Funktion $x\mapsto x^3 - x + \frac 12$ für alle $x\ge 0$ positive Werte hat (denn wir können die Minima dieser Funktion ausrechnen). Mit diesen Methoden ist es dann sogar einfach, die kleinste reelle Zahl $c\ge 0$ auszurechnen, so dass $x^3+c \ge x$ für alle $x\ge 0$ gilt. Wenn Sie sich an diesen Stoff noch aus der Schule erinnern, dann versuchen Sie es doch einmal!


\end{document}
