\documentclass{ximera}
% \input{preamble.tex}
\usepackage[utf8]{inputenc}
\title{Bruchrechnung}
\begin{document}
\begin{abstract}
Wiederholung: Bruchrechnung, mit Übungen.
\end{abstract}
\maketitle

Die Bruchrechenregeln sind (für ganze Zahlen $a,b,c,d\in\mathbb Z$, $b\ne 0$, $d\ne 0$):

Kürzungsregel:
\[
\frac{a}{b} = \frac{ad}{bd}.
\]

Addition:
\[
\frac{a}{b} + \frac{c}{d} = \frac{ad  +cb}{bd}.
\]

Subtraktion:
\[
\frac{a}{b} - \frac{c}{d} = \frac{ad  -cb}{bd}.
\]

Produkt:
\[
\frac{a}{b} \cdot \frac{c}{d} = \frac{ac}{bd}.
\]

Quotient
\[
\frac{a}{b} : \frac{c}{d} = \frac{ad}{bc},\qquad \text{falls } c\ne 0.
\]


In den folgenden Übungsaufgaben sollen in der Antwort immer Zähler und Nenner \emph{der gekürzten Bruchs} angegeben werden.


\begin{question}
Was ist $\frac{1}{6} + \frac{2}{6}$?
\begin{solution}
Die richtige Lösung:

Zähler \answer{$1$}

Nenner \answer{$2$}
\end{solution}
\end{question}


\begin{question}
Was ist $\frac 37 + \frac{39}{7}$?
\begin{solution}
Die richtige Lösung ist
\answer{$6$}.
\end{solution}
\end{question}

\begin{question}
Was ist $\frac{1}{6} + \frac{3}{8}$?
\begin{solution}
Die Lösung ist doch
\answer{$13/24$}.
\end{solution}
\end{question}
\end{document}
