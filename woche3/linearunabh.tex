\documentclass{ximera}
% \input{preamble.tex}
\usepackage[utf8]{inputenc}
\title{Woche 3/Lineare Unabhängigkeit}
\begin{document}
\begin{abstract}
Hier finden Sie Material rund um den Begriff der Linearen Unabhängigkeit.
\end{abstract}
\maketitle

Sei $V$ ein Vektorraum.

Wir haben in der Vorlesung den folgenden Begriff definiert:

\textbf{Definition}
Eine Familie von Vektoren $v_1,\dots, v_n\in V$ heißt \emph{linear unabhängig}, wenn für alle $\lambda_1,\dots, \lambda_n$ mit
\[
\sum_{i=1}^n \lambda_iv_i = \mathbf{0}
\]
gilt: Für alle $i$ ist $\lambda_i = 0$.

Andernfalls heißt die Familie linear abhängig.
\bigskip

In der Vorlesung haben wir auch die folgende Bemerkung gesehen:

\textbf{Satz}
Eine Familie von Vektoren $v_1,\dots, v_n\in V$ ist genau dann linear abhängig, wenn es ein $i$ mit $1\le i\le n$ gibt, so dass sich $v_i$ als Linearkombination der Vektoren $v_1, \dots, v_{i-1}, v_{i+1}, \dots, v_n$ (also: der anderen $n-1$ Vektoren) schreiben lässt.
\bigskip

\textbf{Begründung.} Sind die Vektoren linear abhängig, so gibt es $\lambda_1, \dots, \lambda_n\in \mathbb R$, die nicht alle $=0$ sind, so dass
\[
\lambda_1v_1 + \lambda_2 v_2 + \cdots + \lambda_nv_n = \mathbf{0}.
\]
Sei $i$ zwischen $1$ und $n$, so dass $\lambda_i \ne 0$ --- da nicht alle $\lambda$'s verschwinden, können wir ein solches $i$ finden. Nun subtrahieren wir auf beiden Seiten $\lambda_iv_i$ und teilen durch $\lambda_i$ (hier benutzen wir, dass $\lambda_i\ne 0$). Wir erhalten:
\[
v_i = \sum_{j\ne i} -\frac{\lambda_j}{\lambda_i} v_j,
\]
wobei sich die Summe auf der rechten Seite über alle $j$ zwischen $1$ und $n$ erstreckt, die von $i$ verschieden sind. Damit haben wir gerade $v_i$ als Linearkombination der anderen Vektoren dargestellt.

Ist andersherum
\[
v_i = \sum_{j\ne i} \mu_j v_j,
\]
so erhalten wir mit $\mu_i := -1$ die Darstellung $\sum_{j=1}^n \mu_j v_j = \mathbf{0}$ des Nullvektors als Linearkombination der $v_j$, in der nicht alle Koeffizienten $=0$ sind (denn wir wissen ja, dass wenigstens $\mu_i = -1 \ne 0$).
\bigskip


\begin{question}
\textbf{Beispiel.} Ist $v_1,\dots, v_n\in V$ eine Familie von Vektoren, in der der Nullvektor vorkommt, so ist sie linear abhängig. Begründen Sie das anhand der Definition!
\begin{solution}
\begin{free-response}
Ist etwa $v_i =\mathbf{0}$, so erhalten wir mit $\lambda_i:=1$ und $\lambda_j=0$ für alle $j\ne i$ die Linearkombination
\[
\sum_{j=1}^n \lambda_j v_j = 0v_1 + 0v_2 + \cdots + 1v_i + \cdots + 0v_n = 1v_i = 1\cdot \mathbf{0} = \mathbf{0},
\]
in der nicht alle Koeffizienten $=0$ sind.
\end{free-response}
\end{solution}
\end{question}

\begin{question}
\textbf{Beispiel.} Seien $v, w\in V$, $v\ne \mathbf{0}$. Dann gilt: $v$ und $w$ sind genau dann linear abhängig, wenn $w$ ein Vielfaches von $v$ ist, d.h., wenn sich $w$ in der Form $w= \lambda v$ schreiben lässt. Begründen Sie dies anhand der Definition/des Satzes!
\begin{solution}
    \begin{free-response}
Wenn sich $w$ in der Form $\lambda v$ schreiben lässt, also als ``Linearkombination'' des einen Vektors $v$, so sind die beiden Vektoren $v, w$ nach dem Satz linear abhängig.

Sind andersherum $v$ und $w$ linear abhängig, so gibt es nach Definition reelle Zahlen $\lambda$ und $\mu$ mit
\[
\lambda v + \mu w = \mathbf{0},
\]
und so dass $\lambda\ne 0$ oder $\mu\ne 0$. Wäre aber $\mu = 0$, so erhielten wir $\lambda v = \mathbf{0}$, was wegen $v\ne\mathbf{0}$ nur für $\lambda=0$ möglich wäre --- dann wären aber doch $\lambda=\mu=0$, was wir ausgeschlossen hatten. Also ist $\mu \ne 0$ und wir können tatsächlich $w$ als Vielfaches von $v$ schreiben, nämlich
\[
w = -\frac{\lambda}{\mu} v.
\]
    \end{free-response}
\end{solution}
    
\end{question}

\begin{question}
\textbf{Beispiel.} Wir wollen untersuchen, für welche reellen Zahlen $a$ die Vektoren
\[
v = \left( \begin{array}{c}
1 \\ 3
\end{array} \right), \quad
w = \left( \begin{array}{c}
a \\ 6
\end{array} \right)
\]
linear unabhängig sind.
\begin{solution}
\begin{hint}
Wende das vorherige Beispiel an; dies ist möglich, weil $v\ne\mathbf{0}$.
\end{hint}
\begin{hint}
Wenn $w = \lambda v$ sein soll, muss $\lambda = 3$ sein --- egal, was $a$ ist, denn nur für $\lambda = 3$ ist die erste Zeile richtig.
\end{hint}
\begin{hint}
Damit die zweite Zeile der Gleichung $w = 3 v$ richtig ist, muss $a=2$ sein. Genau in diesem Fall kann man also $w$ als Vielfaches von $v$ schreiben. Mit anderen Worten: Genau in diesem Fall sind die beiden Vektoren linear \emph{abhängig}.
\end{hint}
Die Vektoren sind linear unabhängig für alle $a$ bis auf $a=$ \answer{$2$}.
\end{solution}
\end{question}


\begin{question}
\textbf{Beispiel.}
Seien $v_1, v_2, v_3\in V$ Vektoren mit $v_2 - v_1 = v_3 - v_2$. Dann sind $v_1, v_2, v_3$ linear abhängig. Begründen Sie das!
\begin{solution}
\begin{free-response}
Wenn wir in der obigen Gleichung auf beiden Seiten $v_3$ subtrahieren und $v_2$ addieren, so erhalten wir
\[
-v_1 + 2v_2 - v_3 = \mathbf{0},
\]
und dies ist eine Linearkombination von $v_1$, $v_2$, $v_3$, die den Nullvektor darstellt, und in der nicht alle Koeffizienten $=0$ sind (in diesem Fall ist sogar kein einziger Koeffizient $=0$, aber das spielt eigentlich keine Rolle).
\end{free-response}
\end{solution}
\end{question}


\end{document}
