\documentclass{ximera}
% \input{preamble.tex}
\usepackage[utf8]{inputenc}
\title{Bruchrechnung}
\begin{document}
\begin{abstract}
Wiederholung: Bruchrechnung, mit Übungen.
\end{abstract}
\maketitle

\begin{question}
Was ist $\frac 37 + \frac{39}{7}$?
\begin{solution}
Die richtige Lösung ist
\answer{$6$}.
\end{solution}
\end{question}

\begin{question}
Was ist $\frac{1}{6} + \frac{2}{6}$?
\begin{solution}
Die richtige Lösung ist (als gekürzter Bruch!)
\answer{$\frac{1}{2}$}.
\end{solution}
\end{question}

\begin{question}
Was ist $\frac{1}{6} + \frac{3}{8}$?
\begin{solution}
Die richtige Lösung ist ja wohl
\answer{$\frac{13}{24}$}.
\end{solution}
\end{question}
\end{document}
