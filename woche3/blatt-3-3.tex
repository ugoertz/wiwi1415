\documentclass{ximera}
% \input{preamble.tex}
\usepackage[utf8]{inputenc}
\title{Woche 3 / Blatt 3, Aufgabe 3}
\begin{document}
\begin{abstract}
Hier finden Sie Material zur Aufgabe 3 auf Blatt 3 (Kern einer linearen Abbildung)
\end{abstract}
\maketitle

\textbf{Aufgabe}
Wir betrachten die lineare Abbildung
\[ \phi: \mathbb R^3 \to \mathbb R^2 \quad \mbox{mit} \quad
\phi \left(\begin{array}{c}
x_1\\x_2\\x_3 
\end{array}  \right) = 
\left( \begin{array}{c}
   x_1-x_2\\x_2-x_3 
\end{array}   \right). \]
Bestimmen Sie eine Basis des Kerns $\mathop{\rm Ker}(\phi)$ dieser linearen
Abbildung. Wie gro"s ist die Dimension des Kerns?


\begin{question}
Zuerst benötigen wir die Definition des Kerns einer linearen Abbildung. Wie lautet sie?
\begin{solution}
    \begin{free-response}
        Sei $\phi\colon V \rightarrow W$ eine lineare Abbildung. Dann ist 
\[
\mathop{\rm Ker} \phi = \{ v\in V;\ \phi(v) = \mathbf{0} \}.
\]
    \end{free-response}
\end{solution}
\end{question}


Bestimmen wir als den Kern, also die Menge aller Elemente von $\mathbb R^3$, die unter $\phi$ auf den Nullvektor abgebildet werden. Für 
\[
x= \left(\begin{array}{c}
x_1\\x_2\\x_3 
\end{array}  \right)
\]
gilt $\phi(x) = 0$ genau dann, wenn beide Einträge des Vektors $\phi(x)$ Null sind, also genau dann, wenn  $x_1 - x_2 = 0$ und $x_2 - x_3 = 0$. Mit anderen Worten: $\phi(x) = 0$ gilt genau dann, wenn $x_1 = x_2$ und $x_2 = x_3$, d.h., wenn alle drei Einträge von $x$ übereinstimmen:

\begin{question}
Geben Sie die Menge $\mathop{\rm Ker}\phi$ in Formelschreibweise an!
\begin{solution}
    \begin{free-response}
        Es ist 
\[
\mathop{\rm Ker}\phi = \left\{  \left(
\begin{array}{c}
x \\ x \\ x    
\end{array}
\right)  ;\ x\in \mathbb R  \right\}.
\]
Mit anderen Worten:
\[
\mathop{\rm Ker}\phi = \left\{ x\cdot  \left(
\begin{array}{c}
1 \\ 1 \\ 1    
\end{array}
\right)  ;\ x\in \mathbb R  \right\}.
\]

(Der Kern einer linearen Abbildung ist immer ein Untervektorraum des Definitionsbereichs.)
    \end{free-response}
\end{solution}
\end{question}


\begin{question}
    Als nächstes brauchen wir die Definition des Begriffs der Basis eines Vektorraums. Geben Sie sie an.
\begin{solution}
    \begin{free-response}
Sei $V$ ein Vektorraum. Eine Familie $v_1, \dots, v_n$ von Elementen von $V$ heißt Basis, wenn sich jeder Vektor aus $V$ in eindeutiger Weise als Linearkombination der Vektoren $v_1, \dots, v_n$ darstellen lässt.

(Eine andere Möglichkeit, den Begriff der Basis zu charakterisieren/definieren, ist folgende:
Sei $V$ ein Vektorraum. Eine Familie $v_1, \dots, v_n$ von Elementen von $V$ heißt Basis, wenn $V$ von den Vektoren $v_1,\dots, v_n$ erzeugt wird und die Vektoren $v_1, \dots, v_n$ linear unabhängig sind.)
    \end{free-response}
\end{solution}

\end{question}


Wir haben gesehen, dass der Kern von $\phi$ aus den  Vielfachen des Vektors $\left( \begin{array}{c}
    1 \\ 1\\ 1
\end{array} \right)$ besteht. Andererseits gilt für $x, x'\in \mathbb R$ mit $x\ne x'$, dass
\[
x\cdot  \left(
\begin{array}{c}
1 \\ 1 \\ 1    
\end{array}
\right) \ne  x'\cdot  \left(
\begin{array}{c}
1 \\ 1 \\ 1    
\end{array}
\right) 
\]

Also lässt sich jedes Element des Kerns in eindeutiger Weise als Vielfaches des Vektors 
$\left( \begin{array}{c}
    1 \\ 1\\ 1
\end{array} \right)$ darstellen, oder mit anderen Worten: in eindeutiger Weise als Linearkombination der (einelementigen) Familie
$\left( \begin{array}{c}
    1 \\ 1\\ 1
\end{array} \right)$ von Vektoren darstellen.

Also ist die (einelementige) Familie 
$\left( \begin{array}{c}
    1 \\ 1\\ 1
\end{array} \right)$ eine Basis des Kerns von $\phi$.


\begin{question}
Um die Aufgabe abzuschließen, müssen wir noch die Dimension des Kerns bestimmen. Was ist die Definition der Dimension?
    \begin{solution}
        \begin{free-response}
Die Dimension eines Vektorraums ist die Anzahl der Elemente einer Basis. (Diese Anzahl ist für alle Basen gleich.)
        \end{free-response}
    \end{solution}
\end{question}


Wir sehen also: Der Kern der Abbildung $\phi$ hat Dimension $1$.



Eine allgemeine Methode, den Kern einer linearen Abbildung zu bestimmen, bietet der Gauß-Algorithmus. Sei $\phi$ eine lineare Abbildung $\mathbb R^n\rightarrow \mathbb R^m$, und sei $A$ die zugehörige Abbildungsmatrix (dies ist eine $(m\times n)$-Matrix. Dann ist $\mathop{\rm Ker} \phi$ gleich der Lösungsmenge des linearen Gleichungssystems $Ax=\mathbf{0}$, oder mit anderen Worten: $\mathop{\rm Ker} \phi$ ist genau der Nullraum der Matrix $A$. Diesen kann man mit dem Gauß-Algorithmus bestimmen, indem man die Matrix $A$ in Zeilen-Stufen-Form bringt und dann die Lösungsmenge des homogenen Gleichungssystems mit der Methode aus der Vorlesung ``abliest''.

\begin{question}
Geben Sie die Abbildungsmatrix der Abbildung $\phi$ im obigen Beispiel an:
\begin{matrix-answer}
correctMatrix = [ ['1', '-1', '0' ], [ '0', '1', '-1' ] ]
\end{matrix-answer}
\end{question}


\begin{question}
Diese Matrix ist schon fast in Zeilen-Stufen-Form. Eine Möglichkeit, die Matrix auf Zeilen-Stufen-Form zu bringen ist es, die zweite Zeile zur ersten zu addieren. Wir erhalten dann
\begin{matrix-answer}
correctMatrix = [ ['1', '0', '-1' ], [ '0', '1', '-1' ] ]
\end{matrix-answer}
\end{question}


Die Methode aus der Vorlesung (vom 11.11.) ergibt dann, dass der Vektor $\left(\begin{array}{c}
-1\\-1\\-1  \end{array}\right)$ eine Basis des Kerns bildet.

\end{document}
