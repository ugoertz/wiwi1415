\documentclass{ximera}
% \input{preamble.tex}
\usepackage[utf8]{inputenc}
\title{Woche 4/Zeilen-Stufen-Form}
\begin{document}
\begin{abstract}
Hier finden Sie Material rund um den Begriff der Zeilen-Stufen-Form.
\end{abstract}
\maketitle

Sei $A$ eine $(m\times n)$-Matrix. Wir sagen, die Matrix $A$ sei in Zeilen-Stufen-Form, wenn gilt:
\begin{itemize}
\item
Enthält $A$ eine Nullzeile, so sind auch alle Zeilen darunter Nullzeilen.
\item
In jeder Nicht-Nullzeile ist der erste von Null verschiedene Eintrag eine $1$. Diese wird als ``führende Eins'' der jeweiligen Zeile bezeichnet.
\item
In den Spalten, die eine führende Eins enthalten sind alle anderen Einträge (unter und über der führenden Eins) gleich Null.
\item
Enthalten zwei übereinander liegende Zeilen eine führende Eins, so liegt die der unteren Teile weiter rechts.
\end{itemize}

\textbf{Beispiel.}
Die folgenden Matrizen haben Zeilen-Stufen-Form:
\[
\left(
\begin{array}{cccccc}
0 & 1 & 0 & 0 & 6 & 7 \\
0 & 0 & 1 & 0 & 7 & 0 \\
0 & 0 & 0 & 0 & 0 & 0 \\
0 & 0 & 0 & 0 & 0 & 0 
\end{array}
\right), \quad
%
\left(
\begin{array}{ccccccc}
1 & 0 & 0 & 6 & 0 & 1 \\
0 & 1 & 0 & 7 & 0 & 1 \\
0 & 0 & 1 & 0 & 0 & 1 \\
0 & 0 & 0 & 0 & 1 & 0
\end{array}
\right), \quad
%
\left(
\begin{array}{ccccc}
1 & 0 & 0 & 0 \\
0 & 1 & 0 & 0 \\
0 & 0 & 1 & 0 \\
0 & 0 & 0 & 1 
\end{array}
\right).
\]

Die folgenden Matrizen haben keine Zeilen-Stufen-Form:
\[
\left(
\begin{array}{cccccc}
0 & 0 & 0 & 1 & 6 & 7 \\
0 & 0 & 1 & 0 & 7 & 0 \\
0 & 1 & 0 & 0 & 0 & 0 \\
1 & 0 & 0 & 0 & 0 & 0 
\end{array}
\right), \quad
%
\left(
\begin{array}{ccccccc}
1 & 0 & 0 & 6 & 0 & 1 \\
0 & 0 & 0 & 0 & 0 & 0 \\
0 & 0 & 1 & 0 & 0 & 1 \\
0 & 0 & 0 & 0 & 1 & 0
\end{array}
\right), \quad
%
\left(
\begin{array}{ccccc}
1 & 0 & 0 & 1 \\
0 & 1 & 0 & 0 \\
0 & 0 & 1 & 0 \\
0 & 0 & 0 & 1 
\end{array}
\right).
\]


\begin{question}
Ist die folgende Matrix in Zeilen-Stufen-Form?
\[
\left(
\begin{array}{cccccc}
1 & 0 & 0 & 0 & 6 & 7 \\
0 & 2 & 1 & 0 & 7 & 0 \\
0 & 0 & 0 & 1 & 0 & 0 \\
0 & 0 & 0 & 0 & 0 & 0 
\end{array}
\right)
\]
\begin{solution}
\begin{multiple-choice}
\choice{Ja.}
\choice[correct]{Nein.}
\end{multiple-choice}
\end{solution}
\end{question}


\begin{question}
Ist die folgende Matrix in Zeilen-Stufen-Form?
\[
\left(
\begin{array}{cccccc}
0 & 0 & 1 & 0 & 6 & 0 \\
0 & 0 & 0 & 0 & 0 & 1 \\
0 & 0 & 0 & 0 & 0 & 0 \\
0 & 0 & 0 & 0 & 0 & 0 
\end{array}
\right)
\]
\begin{solution}
\begin{multiple-choice}
\choice[correct]{Ja.}
\choice{Nein.}
\end{multiple-choice}
\end{solution}
\end{question}


\begin{question}
Ist die folgende Matrix in Zeilen-Stufen-Form?
\[
\left(
\begin{array}{cccccc}
1 & 0 & 0 & 0 & 6 & 0 \\
0 & 1 & 1 & 0 & 7 & 0 \\
0 & 0 & 0 & 1 & 0 & 0 \\
0 & 0 & 0 & 0 & 0 & 1 
\end{array}
\right)
\]
\begin{solution}
\begin{multiple-choice}
\choice[correct]{Ja.}
\choice{Nein.}
\end{multiple-choice}
\end{solution}
\end{question}


\begin{question}
Ist die folgende Matrix in Zeilen-Stufen-Form?
\[
\left(
\begin{array}{cccccc}
0 & 1 & 0 & 0 & 0 & 1 \\
1 & 0 & 1 & 0 & 1 & 0 \\
0 & 1 & 0 & 1 & 0 & 1 \\
0 & 0 & 1 & 0 & 1 & 0 
\end{array}
\right)
\]
\begin{solution}
\begin{multiple-choice}
\choice{Ja.}
\choice[correct]{Nein.}
\end{multiple-choice}
\end{solution}
\end{question}


\begin{question}
Bringe die folgende Matrix auf Zeilen-Stufen-Form:
\[
\left(
\begin{array}{ccccccc}
0 & 5 & 10& 1 & 5 & 1 & 20\\
0 & 4 & 8 & 3 & 4 & 1 & 25\\
0 & 4 & 8 & 1 & 4 & 2 & 18\\
0 & 3 & 6 & 1 & 3 & 1 & 14
\end{array}
\right)
\]
\begin{solution}
 \begin{matrix-answer}
correctMatrix = [ [ '0', '1',  '2', '0', '1', '0',  '3' ],
                  [ '0', '0',  '0', '1', '0', '0',  '4' ],
                  [ '0', '0',  '0', '0', '0', '1',  '1' ],
                  [ '0', '0',  '0', '0', '0', '0',  '0' ]
]
\end{matrix-answer}
\end{solution}
\end{question}

\begin{question}
Bringe die folgende Matrix auf Zeilen-Stufen-Form:
\[
\left(
\begin{array}{ccccc}
0 & 2 & 0 & 2\\
3 & 4 & 2 & 0\\
0 & 2 & 2 & 1\\
3 & 3 & 2 & 1
\end{array}
\right)
\]
\begin{solution}
\begin{hint}
Um die Rechnungen zu vereinfachen, ist es manchmal besser, die führenden Einträge erst zum Schluss auf $1$ zu bringen.
\end{hint}
\begin{matrix-answer}
correctMatrix = [ [ '1',  '0', '0',  '0' ],
                  [ '0',  '1', '0',  '0' ],
                  [ '0',  '0', '1',  '0' ],
                  [ '0',  '0', '0',  '1' ]
]
\end{matrix-answer}
\end{solution}
\end{question}

\end{document}
