\documentclass{ximera}
% \input{preamble.tex}
\usepackage[utf8]{inputenc}
\title{Potenzrechnung}
\begin{document}
\begin{abstract}
Wiederholung: Potenzrechnung, mit Übungen.
\end{abstract}
\maketitle

Die Potenzrechenregeln sind (für relle Zahlen $x, y, z\in \mathbb R$ und ganze Zahlen $a,b,c\in\mathbb Z$):

Ist $a\ge 1$ eine natürliche Zahl, so gilt
\[
x^a = x \cdot \cdots \cdot x\qquad \text{(mit $a$ Faktoren auf der rechten Seite)}.
\]

\[
x^0 = 1.
\]

\begin{equation}\label{P1}
x^{a+b} = x^a x^b.
\end{equation}

\[
x^{-1} = \frac{1}{x^a}.
\]

\begin{equation}\label{P2}
(x^a)^b = x^{ab}.
\end{equation}

Schreibweise:
\[
{x^a}^b := x^{(a^b)})
\]



\begin{question}
Was ist $x^3x^2x^{-3}$?
\begin{solution}
\begin{hint}
Die gesuchte Potenz von $x$ berechnet sich durch Addition der Exponenten: $3+2+(-3)=$? (Potenzregel~\eqref{P1}).
\end{hint}
Das Ergebnis ist \answer{x^2}.
\end{solution}
\end{question}


\begin{question}
Was ist $(y^7x^3)^2y$?
\begin{hint}
Die Potenzen von $x$ und $y$ müssen getrennt zusammengefasst werden. Verwende die Regeln~\eqref{P1} und~\eqref{P2}.
\end{hint}
\begin{solution}
Das Ergebnis ist \answer{x^6y^{15}}.
\end{solution}
\end{question}

\begin{question}
Vereinfache den Ausdruck $\frac{x^3y^2(x+y)^2}{xyz}$.
\begin{solution}
Die Lösung ist
\answer{x^4yz^{-1} + 2 x^3y^2z^{-1} + x^2y^3z^{-1}}.
\end{solution}
\end{question}

\begin{question}
Vereinfache den Ausdruck ${x^3}^{-2} x^4.$
\begin{solution}
Das Ergebnis ist \answer{1/x^{2}}
\end{solution}
\end{question}

\begin{question}
Vereinfache den Ausdruck $\frac{x^6z^2}{y^2} + \left(\frac{y}{x^3z}\right)^{-2}$.
\begin{solution}
Das Ergebnis ist \answer{0}
\end{solution}
    
\end{question}
\end{document}
