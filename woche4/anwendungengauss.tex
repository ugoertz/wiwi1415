\documentclass{ximera}
% \input{preamble.tex}
\usepackage[utf8]{inputenc}
\title{Woche 4/Anwendungen des Gauß-Algorithmus}
\begin{document}
\begin{abstract}
Hier finden Sie Material über verschiedene Anwendungen des Gauß-Algorithmus.
\end{abstract}
\maketitle


\textbf{Lösen linearer Gleichungssysteme}

\begin{question}
Sei
\[
A = \left(
\begin{array}{cccccc}
  0&  3&  0&  9& 12&  1\\
  0&  2&  0&  6&  8&  0\\
  0& -3& -2&-13&-12& -1
\end{array}
\right),\qquad
b = \left(
\begin{array}{c}
1 \\ 1 \\ 1
\end{array}
\right)
\]
Wir wollen die Lösungsmenge des linearen Gleichungssystems $Ax=b$ bestimmen.

\begin{solution}
Bringen Sie die erweiterte Koeffizientenmatrix $(A, b)$ auf Zeilen-Stufen-Form.
\begin{matrix-answer}
correctMatrix = [
[   '0',    '1',    '0',    '3',    '4',    '0',  '1/2'],
[   '0',    '0',    '1',    '2',    '0',    '0',   '-1'],
[   '0',    '0',    '0',    '0',    '0',    '1', '-1/2'] ]
\end{matrix-answer}
\end{solution}

\begin{solution}
Lesen Sie aus der Zeilen-Stufen-Form nach der Methode der Vorlesung eine spezielle Lösung des LGS ab:
\begin{matrix-answer}
correctMatrix = [ [ '0'], ['1/2'], ['-1'], ['0'], ['0'], ['-1/2']]
\end{matrix-answer}
\end{solution}

\begin{solution}
Was ist die Dimension des Lösungsraums des zugehörigen homogenen LGS?
\answer{$3$}
\end{solution}

\begin{solution}
Bestimmen Sie nach der Methode der Vorlesung eine Basis des Lösungsraums des zugehörigen homogenen LGS und geben Sie die Basisvektoren hier als Spalten einer Matrix an.
\begin{matrix-answer}
correctMatrix = [
[   '-1',    '0',    '0'  ],
[   '0',    '3',    '4'  ],
[   '0',    '2',    '0'  ],
[   '0',    '-1',    '0'  ],
[   '0',    '0',    '-1'  ],
[   '0',    '0',    '0'  ] ]
\end{matrix-answer}
\end{solution}
\end{question}


\textbf{Kern einer linearen Abbildung}


Sei $\varphi\colon \mathbb R^4 \rightarrow \mathbb R^3$ die lineare Abbildung
\[
\left(
\begin{array}{c}
x_1 \\ x_2 \\ x_3 \\ x_4
\end{array}
\right)
\mapsto
\left(
\begin{array}{c}
x_3 \\ 0 \\ 2x_1 + x_2 - x_4
\end{array}
\right)
\]

\begin{question}
Wir wollen den Kern von $\varphi$ bestimmen.

\begin{solution}
Geben Sie die Abbildungsmatrix $A$ der Abbildung $\varphi$ an. (Erinnerung: Es gilt dann $\varphi(x) = Ax$ für alle $x\in \mathbb R^4$.)
\begin{matrix-answer}
correctMatrix = 
[ [ '0','0','1','0'], ['0','0','0','0'], ['2','1','0','-1']]
\end{matrix-answer}
\end{solution}

\begin{solution}
Bringen Sie diese Matrix auf Zeilen-Stufen-Form.
\begin{matrix-answer}
correctMatrix = [
[ '1', '1/2', '0', '-1/2' ], [ '0', '0', '1', '0' ], [ '0', '0', '0', '0' ]
]
\end{matrix-answer}
\end{solution}

\begin{solution}
Es gilt $x\in \mathop{\rm Ker}\varphi \Leftrightarrow \varphi(x) = 0 \Leftrightarrow Ax=0$, also ist $\mathop{\rm Ker}\varphi$ die Lösungsmenge des homogenen LGS $Ax=0$. Geben Sie eine Basis der Lösungsmenge mit der Methode der Vorlesung an. Geben Sie die Basisvektoren als Spalten einer Matrix ein.
\begin{matrix-answer}
correctMatrix = [
[  '1/2', '-1/2' ],
[  '-1', '0' ],
[  '0', '0' ],
[  '0', '-1' ],
] 
\end{matrix-answer}
\end{solution}
\end{question}


\textbf{Rang einer Matrix}

Auch um den Rang einer Matrix zu bestimmen, kann man den Gauß-Algorithmus benutzen. Dies liegt an den folgenden beiden Tatsachen:
\begin{enumerate}
\item
Unter den Umformungen des Gauß-Algorithmus ändert sich der Rang einer Matrix nicht. (Dies ist auch richtig für die entsprechenden Spaltenumformungen --- solange es nur um den Rang geht, darf man also auch Spaltenumformungen vornehmen, wenn das die Rechnung vereinfacht.)
\item
Ist $A$ eine Matrix in Zeilen-Stufen-Form, so ist der Rang von $A$ gerade die Anzahl der Nicht-Null-Zeilen (mit anderen Worten: die Anzahl der führenden Einsen).)
\end{enumerate}

\begin{question}
    Bestimme den Rang der Matrix
\[
\left(
\begin{array}{cccc}
 4&3&2&1  \\
 6&5&4&3  \\
 8&7&6&4  \\
 1&2&3&4   
\end{array}
\right)
\]

\begin{solution}
Bringe die Matrix $A$ durch Zeilenumformungen auf Zeilen-Stufen-Form!
\begin{matrix-answer}
correctMatrix = [ [ '1',  '0', '-1',  '0'],
[ '0',  '1',  '2',  '0'],
[ '0',  '0',  '0',  '1'],
[ '0',  '0',  '0',  '0'] ]
\end{matrix-answer}
(Um den Rang zu bestimmen, wären auch Spaltenumformungen zulässig, aber dann ist die Zeilen-Stufen-Form nicht eindeutig, so dass das Ergebnis hier nicht abgefragt werden könnte.)    
\end{solution}

\begin{solution}
Was ist der Rang der Matrix $A$? \answer{$3$}
\end{solution}
\end{question}


\textbf{Inverse einer Matrix}

Wie in der Vorlesung besprochen, kann man mit dem Gauß-Algorithmus die Inverse einer regulären Matrix ausrechnen.

\begin{question}
    Berechnen Sie die Inverse der Matrix
\[
A = \left(
\begin{array}{ccc}
2 & 0 & 1 \\
0 & 1 & 1 \\
3 & 2 & 1
\end{array}
\right)
\]

\begin{solution}
    \begin{matrix-answer}
correctMatrix = [
[ '1/5', '-2/5',  '1/5'],
['-3/5',  '1/5',  '2/5'],
[ '3/5',  '4/5', '-2/5']
]
    \end{matrix-answer}
\end{solution}
Beim Berechnen der Inversen sollten Sie immer nachher die Probe machen (das Matrizenprodukt $AA^{-1}$ muss die Einheitsmatrix ergeben!), um sicherzugehen, dass Sie keine Rechenfehler gemacht haben.
\end{question}

\textbf{Test auf lineare Unabhängigkeit}

Mit dem Gauß-Algorithmus können Sie überprüfen, ob Vektoren $v_1, \dots, v_n \in\mathbb R^m$ linear unabhängig sind. Dies ist nämlich genau dann der Fall, wenn die $(m\times n)$-Matrix, deren Spalten die Vektoren $v_i$ bilden, Rang $n$ hat.

\begin{question}
    Überprüfen Sie, ob die folgenden Vektoren linear unabhängig sind:
\[
v_1 = \left(
\begin{array}{c}
   2 \\ 0 \\ 1 \\ 0 \\ 1
\end{array}
\right), \quad
v_2 = \left(
\begin{array}{c}
   1 \\ 3 \\ 1 \\ 1 \\ 0
\end{array}
\right), \quad
v_3 = \left(
\begin{array}{c}
   4 \\ 1 \\ 1 \\ 2 \\ 0
\end{array}
\right), \quad
v_4 = \left(
\begin{array}{c}
   1 \\ 1 \\ 1 \\ 3 \\ 0
\end{array}
\right).
\]
\begin{solution}
\begin{multiple-choice}
\choice{Die Vektoren sind linear abhängig.} 
\choice[correct]{Die Vektoren sind linear unabhängig.}
\end{multiple-choice}
\end{solution}
\end{question}


\textbf{Berechnen der Determinante}

Die Umformungen des Gauß-Algorithmus ändern zwar (teilweise) die Determinante einer Matrix, aber in kontrollierter Weise. Weil eine Matrix in Zeilen-Stufen-Form automatisch eine obere Dreiecksmatrix ist, kann man ihre Determinante direkt ablesen. Deswegen lässt sich der Gauß-Algorithmus benutzen, um die Determinante einer Matrix zu bestimmen.

\begin{question}
    Bestimmen Sie die Determinante der Matrix
\[
A = \left(
\begin{array}{cccc}
 -1&1&1&1\\
 -2&1&4&1\\
 0&3&1&1 \\
 0&1&-2&3    
\end{array}
\right)
\]
\begin{solution}
Es ist $\det(A) =$ \answer{$14$}
\end{solution}
\end{question}

\begin{question}
    Bestimmen Sie die Determinante der Matrix
\[
B = \left(
\begin{array}{ccccc}
1&0&-2&3&1 \\
0&-3&-2& 1/2& 0 \\
1/3& 1/3& 1/3& -2/3& 5/3 \\
2&1&2&1&2 \\
0&2&1&1&0  
\end{array}
\right)
\]
\begin{solution}
Es ist $\det(B) =$ \answer{$-22$}
\end{solution}
\end{question}




\end{document}
